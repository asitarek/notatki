\documentclass{report}

\usepackage[cp1250]{inputenc}
\usepackage[polish]{babel}
\usepackage{polski}
\usepackage[T1]{fontenc}

\begin{document}

Definicja szeregu czasowego

Funkcja warto�ci oczekiwanej

$$
\mu_t = E(X_t)
$$

Funkcja autokowariancji

$$
\gamma(t, h) = Cov(X_t, X_{t+h}) = E[(X_t - \mu_t)(X_{t+h} - \mu_{t+h})]
$$

Funkcja autokorelacji

$$
\rho(t, h)  = Corr(X_t, X_{t+h})
$$

Funkcja cz�ciowej autokorelacji


Szereg czasowy nazywamy stacjonarnym, je�eli


Proces stacjonarny w szerszym sensie
Proces stacjonarny w w�szym sensie	

Test Dickeya-Fullera
Rozszerzony test Dickeya-Fullera
Test Philipsa-Perrona
Test KPSS

Operator przesuni�cia

Szereg czasowy jest zintegrowany w stopniu d


Proces �redniej ruchomej rz�du q

$$
X_t = \epsilon_t + \theta_1 \epsilon_{t-1} + \cdots + \theta_q \epsilon_{t-q}
$$

Proces autoregresyjny rz�du p

$$
X_t = \phi_1 \epsilon_{t-1} + \cdots + \phi_p X_{t-p} + \epsilon_t
$$

Proces ARMA rz�du p i q

$$
X_t = \phi_1 \epsilon_{t-1} + \cdots + \phi_p X_{t-p} + \epsilon_t + \theta_1 \epsilon_{t-1} + \cdots + \theta_q \epsilon_{t-q}
$$

Proces ARIMA rz�du p, d i q

Wyg�adzanie wyk�adnicze

Dekompozycja Wolda

R�wnania Yule'a-Walkera

Kryteria informacyjne

Sezonowe modele SARIMA

Test Q Ljunga-Boxa

Statystyka Durbina-Watsona

Test Jarque-era


\end{document}